\pagestyle{myheadings}
\markboth{CCD Photometry}{CCD Photometry}
\setcounter{page}{1}

\begin{center}
{\Huge\bf CCD photometry}
\end{center}

{\bf \large{1. Introduction}}

This exercise involves the analysis of two images of a cluster of
galaxies which have been obtained using a
$v-$filter ($\lambda_{centre}=5500$\AA) and an $i-$filter ($\lambda_{centre}=8000$\AA). 
The first stage in the process is to use images of a standard star
taken through the $v-$filter and $i-$filter at two different airmasses
(elevations) to compute the correction due to atmospheric
absorption and to calculate the zero-point necessary to convert
instrumental magnitudes onto a standard magnitude system. The second stage is to accurately measure
the $v-$ and $i-$magnitudes of the galaxies in the cluster images and
to compute their $(v-i)$ colours. Based on their $(v-i)$
colours the aim of the exercise is to determine which of the
galaxies are likely cluster members, and which are simply
foreground galaxies which just happen to lie along the same line of sight.

\noindent
In your home directory there should be six files present, namely:

\verb,cluster_v.sdf,\\
\verb,cluster_i.sdf,\\
\verb,standard_v_a.sdf,\\
\verb,standard_v_b.sdf,\\
\verb,standard_i_a.sdf,\\
\verb,standard_i_b.sdf,

The first two files are the images of the galaxy cluster taken through
the $v-$ and $i-$filters (.sdf simply indicates that the images are stored in
a particular image format). The other four files are images of the
same standard star taken through the $v-$ and $i-$filters at two
different values of airmass; $\sec \theta=1.0$ (a) and $\sec \theta=1.5$ (b).
Your first task is to use the observations of the
standard star to workout the absorption of the atmosphere in the $v-$
and $i-$filters as a function of airmass ($\sec \theta$).

{\bf \large{2. Standard star analysis}}

All of the photometry measurements in this exercise are going to be
performed using a software package called {\sc gaia} which allows you 
to display, manipulate and analyse two dimensional astronomical images. The first thing to
try is to display one of the standard star images so, for example, type:

{\tt \verb,gaia standard_i_a.sdf &,}

which should load a {\sc gaia} window showing an image of the standard
star (see Fig 1). The basic operation of {\sc gaia} is all ``point and
click'' and is fairly intuitive (online help is available by
clicking on the {\sc help} tab in the top right corner of the {\sc
gaia} window). You should spend some time familiarizing yourself with
how {\sc gaia} operates, learning how to zoom in and out, change the
display levels etc. If you are having difficulties, ask your lab
demonstrator to show you how the basic features work.

\newpage

\begin{figure}
\centerline{\psfig{file=fig1.ps,width=8.0cm,angle=0}}
\caption{{\sc gaia} window displaying the image of a standard star.}
\end{figure}

{\large{\bf 3. Instrumental magnitudes}}

\begin{figure}
\centerline{\psfig{file=fig2.ps,width=8.0cm,angle=0}}
\caption{{\sc gaia} window displaying the image of a standard star
with a circular aperture overlaid (inner circle). The annulus defined
by the two outer circles is used by {\sc gaia} to measure the data
counts coming from the sky background.}
\end{figure}
The first measurements you need to make are to determine the
instrumental magnitudes of the standard stars. Instrumental magnitudes
are simply $m = -2.5\log_{10} C$; where $C$ is the data-counts from
the object, which are proportional to the object's
flux\footnote{instrumental magnitudes are so-called because they
are specific to a particular instrument/telescope combination}. 

When you have one of the standard star images displayed 
in {\sc gaia}, select the ``Image-Analysis'' tab located at
the top of the {\sc gaia} window, and then select ``Aperture photometry''
and ``Results in magnitudes'' from the sub-menus which appear. 
Before measuring any magnitudes it is necessary to set 
the zero-point parameter (top of aperture photometry window) to zero. 
This tells {\sc gaia} that we don't know the proper zero-point and
that we are therefore calculating instrumental magnitudes. 

To measure a magnitude, click the ``define object aperture'' tab and
then, with the mouse, drag out a circular aperture which fully
encloses the flux of the standard star. {\sc gaia} will then automatically define
an annulus enclosed by two yellow circles, within which it will
measure the data counts coming from the sky background (see Fig 2). Now click on the
``calculate results'' tab to calculate the magnitude of the star within your chosen aperture. 
{\sc gaia} will now calculate the instrumental magnitude and the
appropriate magnitude error. {\sc gaia} will also report the ``Sum in
aperture'', which is the data counts within your aperture coming from
the star (i.e. total data counts minus the sky background. You should
check that the magnitude value is simply $-2.5\log_{10} C$; where
$C$ is the ``Sum in aperture''). Note down the data counts, magnitude
and magnitude error. Repeat this process for all four standard star images.
{\bf To obtain the most accurate results it is best to employ the same
sized apertures on each standard star image.}


\newpage
{\bf In your report you should demonstrate with an example that you
understand how {\sc gaia} calculates the magnitude errors (see notes
at the end of this section).}



{\large{\bf 4. Correcting for atmospheric extinction}}

At any airmass ($\sec \theta$) the instrumental $v-$magnitude is:
\[   m_v({\rm at\ airmass\ }\sec \theta) = m_v(0) + e_v\sec \theta   \]
where $m_v(0)$ is the $v-$filter instrumental magnitude that would be obtained if 
observing from above the atmosphere. The parameter $e_v$ is referred
to as the extinction coefficient, and describes how much the observed
instrumental magnitude is affected by atmospheric absorption as a
function of airmass ($\sec \theta$).

The key to determining the extinction coefficient is to observe the
{\it same} star at two different airmasses. You should be able to show
that for two stars observed with the $v-$filter at two different
airmasses, it follows from the above equation that:
\[   e_v = \frac{m_v({\rm at\ airmass\ }\sec \theta_2)-m_v({\rm at\
airmass\ }\sec \theta_1)}{\sec \theta_2 - \sec \theta_1}   \]
and similarly for $e_i$. Use this equation to calculate the values of $e_v$ and $e_i$ and
estimate the corresponding errors (you can assume that the values of
$\sec \theta$ carry negligible uncertainty).

{\large{\bf 5. Converting from instrumental to standard magnitudes}}

Armed with knowledge of the atmospheric absorption as a function of
airmass it is possible to convert observed magnitudes to ``above
atmosphere'' values. It is known from previous observations that on the
standard magnitude system the standard star has above atmosphere
magnitudes of $v=17.700 \pm 0.005$ and $i=17.500\pm 0.005$. The $v-$filter
and $i-$filter images of the galaxy cluster were obtained at zenith 
($\sec \theta=1$), the same airmass as \verb,star_v_a.sdf and star_i_a.sdf,.
Use these two pieces of information to calculate the zero points
necessary to convert from instrumental magnitudes to above atmosphere
magnitudes on the standard magnitude system for the cluster galaxy
images; i.e find $ZP$:
\[
m_{stan} = -2.5\log_{10} C  + ZP 
\]
where $m_{stan}$ is the above atmosphere magnitude of the standard star.

{\large{\bf 6. Determining cluster membership}}

Now that you have calculated the appropriate zero points for the
cluster galaxy images we are in a position to measure magnitudes and
$(v-i)$ colours of the potential cluster galaxies on the standard
magnitude system. The procedure is very straightforward, simply
display the $v-$filter cluster image in {\sc gaia} and measure the
magnitudes, magnitude errors and $(x,y)$ positions of each
galaxy. Repeat the process for the $i-$filter image and produce a
table which lists the $(x,y)$ position, $v-$mag, $v-$mag error,
$i-$mag, $i-$mag error, $(v-i)$ and $(v-i)$ error for each
galaxy. 

{\bf When making these measurements you should think about the
importance of aperture-size choice on the $v-$ and $i-$filter images.}

Once you have your table of results, the next stage is to plot 
$(v-i)$ colour versus $i-$magnitude. You are entirely free to
produce this plot using any software you are familiar with, but 
for convenience a help sheet has been included in the appendix 
describing how to produce suitable plots with a software package
called Gnuplot\footnote{http://www.gnuplot.info/}.

It is known from previous observations that galaxies which lie at 
the centre of clusters tend to be massive and largely populated by old stars
which formed many Gyrs ago. As a result of their old age, galaxies at
the centre of clusters tend to have a red $(v-i)$ colours; i.e
$(v-i)\geq1.5$. Consequently, any galaxy significantly bluer that
$(v-i)=1.5$ is likely to be an unrelated foreground galaxy.
Use this information to decide which of the galaxies are cluster 
members, and which are foreground interlopers.

Re-plot the $(v-i)$ versus $i-$magnitude diagram for just those galaxies which
are likely cluster members. Does there appear to be a relationship
between $(v-i)$ colour and $i-$magnitude? Due to the way galaxies
evolve in cluster environments, it is known observationally
that there is a relationship between the brightness of a cluster
galaxy and it's colour, with brighter galaxies having redder colours
(the so-called ``red sequence''). Using Gnuplot (or any other software you choose) fit a
function of the form:\\
\begin{center}
$(v-i)$=$a\times i-$mag+$b$ 
\end{center}
\noindent
to your data and produce a final plot of $(v-i)$ versus $i-$magnitude with your
best-fitting function overplotted. Does your data display a convincing
red sequence?
